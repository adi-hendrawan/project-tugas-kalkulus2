\documentclass{beamer}

\usepackage[utf8]{inputenc}
\usepackage[T1]{fontenc}
\usepackage{lmodern}
\usetheme{metropolis}

\title{Materi Presentasi: Fungsi Transenden dalam Kalkulus}
\subtitle{Latihan Soal dan Jawaban}
\author{Kadek Adi Hendrawan | \and Rahman Ma'arif Gunawan | \and Reyza Miftah Al Fauzan | \and Salsabilla Fathinah}
\institute{Sekolah Tinggi Teknologi Bandung}
\date{\today}

\begin{document}

\begin{frame}
  \titlepage
\end{frame}

\begin{frame}{Pendahuluan}
  Halo Semua, perkenalkan kami adalah kelompok 2 dari kelas TIRP22A. Pada presentasi kali ini, kami akan membahas tentang fungsi transenden dalam kalkulus.

  \vspace{12pt} % Atur spasi vertikal di sini (misalnya 12pt)
\end{frame}

\section{Definisi Fungsi Transenden}

\begin{frame}{Definisi Fungsi Transenden}
  Fungsi transenden adalah fungsi matematika yang tidak dapat dinyatakan dalam bentuk akar persamaan polinomial dengan koefisien rasional. Beberapa contoh fungsi transenden yang terkenal adalah:

  \begin{itemize}
    \item Fungsi eksponensial: $f(x) = e^x$
    \item Fungsi logaritma: $f(x) = \ln(x)$
    \item Fungsi trigonometri: $f(x) = \sin(x), \cos(x), \tan(x)$
    \item Fungsi hiperbolik: $f(x) = \sinh(x), \cosh(x), \tanh(x)$
    \item Fungsi eksponensial umum: $f(x) = a^x$, dengan $a > 0$ dan $a \neq 1$
  \end{itemize}
\end{frame}

\section{Fungsi Hiperbolik}

\begin{frame}{Fungsi Hiperbolik}
  Fungsi hiperbolik adalah fungsi yang mirip dengan fungsi trigonometri, tetapi menggunakan fungsi eksponensial. Beberapa fungsi hiperbolik yang terkenal adalah:

  \begin{itemize}
    \item Fungsi sinus hiperbolik: $f(x) = \sinh(x)$
    \item Fungsi kosinus hiperbolik: $f(x) = \cosh(x)$
    \item Fungsi tangen hiperbolik: $f(x) = \tanh(x)$
  \end{itemize}
\end{frame}

\begin{frame}{Persamaan Identitas Fungsi Hiperbolik}
  Fungsi hiperbolik juga memiliki persamaan identitas yang mirip dengan fungsi trigonometri:

  \begin{align*}
    \cosh^2(x) - \sinh^2(x) &= 1 \\
    \cosh(x) + \sinh(x) &= e^x \\
    \tanh(x) &= \frac{\sinh(x)}{\cosh(x)}
  \end{align*}
\end{frame}

\section{Latihan Soal}

\begin{frame}{Latihan Soal}
  \begin{enumerate}
    \item Hitunglah nilai dari $e^{\pi}$.
    \item Tentukan nilai dari $\ln(1)$.
    \item Hitunglah nilai dari $\sinh(2)$.
    \item Tentukan nilai integral $\int_{0}^{\ln(2)} \cosh(x) \, dx$.
  \end{enumerate}
\end{frame}

\begin{frame}{Cara Menjawab - Soal 1}
  Soal 1: Hitunglah nilai dari $e^{\pi}$.

  Cara menjawab:
  \begin{itemize}
    \item Gunakan kalkulator atau perangkat lunak matematika yang memiliki fungsi eksponensial.
    \item Ketikkan $e$ diikuti dengan operator pangkat ($\wedge$ atau \texttt{^}) dan kemudian $\pi$.
    \item Tekan tombol \textbf{Hitung} atau tekan \textbf{Enter} untuk mendapatkan hasilnya.
  \end{itemize}
\end{frame}

\begin{frame}{Jawaban - Soal 1}
  Soal 1: Hitunglah nilai dari $e^{\pi}$.

  Jawaban: Nilai dari $e^{\pi}$ adalah $\approx 23.14069$.
\end{frame}

\begin{frame}{Cara Menjawab - Soal 2}
  Soal 2: Tentukan nilai dari $\ln(1)$.

  Cara menjawab:
  \begin{itemize}
    \item Gunakan kalkulator atau perangkat lunak matematika yang memiliki fungsi logaritma alami (natural logarithm).
    \item Ketikkan $\ln$ diikuti dengan tanda kurung buka, angka 1, dan tanda kurung tutup.
    \item Tekan tombol \textbf{Hitung} atau tekan \textbf{Enter} untuk mendapatkan hasilnya.
  \end{itemize}
\end{frame}

\begin{frame}{Jawaban - Soal 2}
  Soal 2: Tentukan nilai dari $\ln(1)$.

  Jawaban: Nilai dari $\ln(1)$ adalah $0$.
\end{frame}

\begin{frame}{Cara Menjawab - Soal 3}
  Soal 3: Hitunglah nilai dari $\sinh(2)$.

  Cara menjawab:
  \begin{itemize}
    \item Gunakan kalkulator atau perangkat lunak matematika yang memiliki fungsi sinus hiperbolik.
    \item Ketikkan $\sinh$ diikuti dengan tanda kurung buka, angka 2, dan tanda kurung tutup.
    \item Tekan tombol \textbf{Hitung} atau tekan \textbf{Enter} untuk mendapatkan hasilnya.
  \end{itemize}
\end{frame}

\begin{frame}{Jawaban - Soal 3}
  Soal 3: Hitunglah nilai dari $\sinh(2)$.

  Jawaban: Nilai dari $\sinh(2)$ adalah $\approx 3.62686$.
\end{frame}

\begin{frame}{Cara Menjawab - Soal 4}
  Soal 4: Tentukan nilai integral $\int_{0}^{\ln(2)} \cosh(x) \, dx$.

  Cara menjawab:
  \begin{itemize}
    \item Gunakan teknik integral untuk menyelesaikan integral tersebut.
    \item Terapkan aturan integral $\int \cosh(x) \, dx = \sinh(x) + C$, di mana $C$ adalah konstanta.
    \item Evaluasi fungsi integral $\sinh(x)$ di batas atas dan batas bawah, kemudian kurangkan hasilnya.
  \end{itemize}
\end{frame}

\begin{frame}{Jawaban - Soal 4}
  Soal 4: Tentukan nilai integral $\int_{0}^{\ln(2)} \cosh(x) \, dx$.

  Jawaban: Nilai integral $\int_{0}^{\ln(2)} \cosh(x) \, dx$ adalah $\approx 1.1752$.
\end{frame}

\section{Penerapan Fungsi Transenden}

\begin{frame}{Penerapan Fungsi Transenden}
  Fungsi transenden memiliki banyak penerapan dalam berbagai bidang, seperti:

  \begin{itemize}
    \item Matematika: Digunakan dalam perhitungan trigonometri, integral, dan diferensial.
    \item Fisika: Muncul dalam persamaan-persamaan fisika yang menjelaskan fenomena alam, seperti hukum gerak Newton dan persamaan Maxwell.
    \item Teknik: Digunakan dalam pengolahan sinyal, pemrosesan gambar, dan komputasi numerik.
  \end{itemize}
\end{frame}

\begin{frame}{Kesimpulan}
  Terima kasih telah membaca presentasi kami. Apakah ada pertanyaan?
\end{frame}

\end{document}
