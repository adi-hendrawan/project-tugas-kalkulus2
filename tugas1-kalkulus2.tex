\documentclass{beamer}
\usetheme{Madrid}
\usecolortheme{beaver}
\usefonttheme{serif}
\usepackage{amsfonts}
\title{Tugas Kalkulus}
\institute{STTB}
\author{Kadek Adi Hendrawan\thanks{NPM:22262011190}}
%\date{06 June 2023} 

\begin{document}
\begin{frame}
\titlepage
\end{frame}

\begin{frame}
\frametitle{Fungsi Eksponen}
Fungsi eksponen adalah fungsi matematika yang menggambarkan hubungan antara bilangan yang dipangkatkan dengan eksponen yang diberikan. Fungsi eksponen dapat dituliskan dengan notasi f(x) = ax, di mana a adalah basis dari fungsi tersebut dan x adalah eksponen yang diberikan.


\end{frame}

\begin{frame}
\frametitle{Eksponensial}
\begin{enumerate}
    \item[]
$ \frac{dy}{dt} $ $ \sim$ y
    \item[]
$ \frac{dy}{dt} =ky$
    \item[] 
$ \frac{dy}{y} =k dt$
    \item[] 
$ \int \frac{dy}{y} =\int k dt$
    \item[]
$ ln \hspace{5pt} y =kt+C$
    \item[] 
$ ln \hspace{5pt} y =kt+ln \hspace{5pt} y_0$
\end{enumerate}
\begin{wrapfigure}{r}{0.5\textwidth} 
  \centering
  \includegraphics[width=0.4\textwidth]{gambar1.jpg}
  \label{fig:contoh gambar 1}
\end{wrapfigure}
\end{frame}
\begin{frame}{Sifat-sifat Eksponen}
    \begin{wrapfigure}{r}{0.7\textwidth}
  \centering
  \includegraphics[width=0.6\textwidth]{gambar2.png}
  \label{fig:contoh gambar 2}
\end{wrapfigure}
\end{frame}

\begin{frame}{Contoh Soal 1}
Tentukan $3^{x+1} = 9^{x-2}$
Soal tersebut menggunakan persamaan Jika a(fx) = 1, maka f(x)=0 dengan $a > 0$ dan $a \neq 1$
\vspace{20pt}
\begin{enumerate}
    \item[]
\begin{tex}
    Jawaban:
\end{tex}
$\log_3(3^{x+1}) = \log_3(9^{x-2})$
    \item[]
\hspace{55pt}$(x+1) \log_3(3) = (x-2) \log_3(9)$
$\log_3(3) = 1$ dan $\log_3(9) = 2$
$x+1 = 2(x-2)$
Selesaikan persamaan linier tersebut:
$x+1 = 2x-4$
$2x - x = 1 + 4$
$x = 5$
Jadi, solusi persamaan $3^{x+1} = 9^{x-2}$ adalah $x = 5$.
    \item[]
\end{enumerate}
\end{frame}

\begin{frame}{Contoh Soal 2}
Tentukan penyelesaian dari persamaan eksponen $(x-2)^{x2-2x} = (x-2)^{x+4}$ Berdasarkan persamaan eksponen di atas didapat dari 4 kondisi berikut.

\begin{enumerate}
    \item[]
\begin{tex}
    Jawaban:
\end{tex}
1. Kondisi 1: $(x-2) \neq 0$
   Jika $(x-2) \neq 0$, maka kita dapat membagi kedua sisi persamaan dengan $(x-2)$:
   $(x-2)^{x^2-2x} \div (x-2) = (x-2)^{x+4} \div (x-2)$
   Karena $(x-2) \neq 0$, kita dapat menyederhanakan persamaan menjadi:
   $(x-2)^{x^2-2x-1} = (x-2)^{x+3}$
   Dalam kondisi ini, untuk memenuhi persamaan, eksponen di kedua sisi persamaan harus sama, yaitu $x^2-2x-1 = x+3$. Dengan menyelesaikan persamaan kuadrat ini, kita dapat menentukan nilai x yang memenuhi kondisi ini.
\end{enumerate}
\end{frame}

\begin{frame}{Kondisi 2 Jawaban 2}
2. Kondisi 2: $(x-2) = 0$
   Jika $(x-2) = 0$, maka $x = 2$. Dalam kasus ini, kita harus memeriksa apakah nilai $x = 2$ memenuhi persamaan asli atau tidak.
\end{frame}
\begin{frame}{Kondisi 3 Jawaban 2}
3. Kondisi 3: $(x-2) > 0$
   Jika $(x-2) > 0$, maka kita dapat menyederhanakan persamaan menjadi:
   $(x-2)^{x^2-2x} = (x-2)^{x+4}$
   Dalam kondisi ini, untuk memenuhi persamaan, eksponen di kedua sisi persamaan harus sama, yaitu $x^2-2x = x+4$. Dengan menyelesaikan persamaan kuadrat ini, kita dapat menentukan nilai x yang memenuhi kondisi ini.
\end{frame}
\begin{frame}{Kondisi 4 Jawaban 2}
4. Kondisi 4: $(x-2) < 0$
   Jika $(x-2) < 0$, maka kita dapat mengubah tanda ketidaksamaan dan membagi kedua sisi persamaan dengan $(x-2)$:
   $(x-2)^{x^2-2x} \div (x-2) = (x-2)^{x+4} \div (x-2)$
   Karena $(x-2) < 0$ dan kita membagi kedua sisi dengan bilangan negatif, tanda ketidaksamaan akan berbalik:
   $[(x-2)^{x^2-2x}] \div (-(x-2)) = [(x-2)^{x+4}] \div (-(x-2))$
   Persamaan ini dapat disederhanakan menjadi:
   $-(x-2)^{x^2-2x-1} = -(x-2)^{x+3}$
   Dalam kondisi ini, untuk memenuhi persamaan, eksponen di kedua sisi persamaan harus sama, yaitu $x^2-2x-1 = x+3$. Dengan menyelesaikan persamaan kuadrat ini, kita dapat.

\end{frame}
\begin{frame}
    \begin{center}
        Thank You
    \end{center}
\end{frame}

\end{document}


